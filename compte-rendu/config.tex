\usepackage{svg}
\usepackage{tikz}
\usepackage[utf8]{inputenc}
\usepackage{lipsum}
\usepackage[T1]{fontenc}
\usepackage{xcolor}
\usepackage[osf]{libertinus}
\usepackage{fancyhdr}
\usepackage[a4paper,top=2.5cm,bottom=2.5cm,left=2cm,right=2cm]{geometry}
\usepackage{lastpage}
\usepackage{titlesec}
\usepackage{tocloft}
\usepackage[hidelinks]{hyperref}
\usepackage{graphicx}
\usepackage{float}
\usepackage{listings}
\usepackage{ulem}
\usepackage{etoolbox}
\usepackage{hyperref}
\usepackage{amsmath}
\usepackage{lastpage}

\usepackage{cite}  

\usetikzlibrary{positioning}

% Déclarations de variables
\renewcommand{\bibname}{\centering Bibliographie}
\newcommand{\projet}{Projet d'algorithmique et de programmation n°1}
\newcommand{\prenomnomA}{Mélissa Colin}
\newcommand{\prenomnomB}{Benjamin Chew}
\newcommand{\referent}{Amina Guermouche}
\newcommand{\responsable}{David Renault}
\newcommand{\nomprojet}{Bourgarde}
\setsvg{inkscapeexe=inkscape}

% Définition de la couleur des barres
\definecolor{enseirb}{RGB}{0, 117, 191}
\definecolor{darkgreen}{RGB}{0, 100, 0}

% Configuration verbatim
\lstdefinestyle{customstyle}{
    language=,
    basicstyle=\ttfamily\small,       
    keywords={pour,si,fin,tant, allant, de, retourner, a, que, sinon}, 
    keywordstyle=\color{red},         
    keywords=[2]{cree_une_position, position_libre, lister_voisins, est_mine, batiment_id, recup_ressource, lister_positions_libres , calculer_ressources, creer_candidat, deplacer_curseur, plus_grand_ou_incomp, ajouter_candidat, selectionner_positions_optimales, calculer_ressources_candidat},    
    keywordstyle=[2]\color{blue},       
    keywords=[3]{et, fonction},
    keywordstyle=[3]\color{black},          
    identifierstyle=\color{gray},     
    moredelim=**[is][\color{darkgreen}]{'}{'}, 
    frame=single,                     
    rulecolor=\color{black},          
    breaklines=true,                  
    postbreak=\mbox{\textcolor{red}{$\hookrightarrow$}\space}, % Add a symbol at the break
    framexleftmargin=5pt,
    framextopmargin=5pt,
    framexbottommargin=5pt,
    framexrightmargin=5pt,
    frameround=tttt % Rounded corners
}

% Configuration de l'en-tête et du pied de page
\pagestyle{fancy}
\fancyhf{}

% En-tête
\fancyhead[L]{\prenomnomA}
\fancyhead[C]{\textbf{\nomprojet}}
\fancyhead[R]{\prenomnomB}

% Pied de page
\fancyfoot[C]{\color{black}{\thepage\ /\ \pageref{LastPage}}}

% Supprime les lignes par défaut des en-têtes et pieds de page
\renewcommand{\headrule}{
    \vspace{5pt}
    \color{enseirb}\hrule width\headwidth height\headrulewidth
    \vspace{5cm}
}
\renewcommand{\footrule}{
    \vspace{18pt}
    \color{enseirb}\hrule width\headwidth height\footrulewidth
    \vspace{5pt}
}
\renewcommand{\footrulewidth}{0.4pt}

% Configuration des titres
\titleformat{\section}[block]
    {\normalfont\huge\bfseries\centering}{\thesection}{1em}{}
\titleformat{\subsection}[block]
    {\normalfont\Large\bfseries}{\thesubsection}{1em}{}

% Configuration de la table des matières
\renewcommand{\cftsecfont}{\bfseries\large}
\renewcommand{\cftsubsecfont}{\normalsize}
\setlength{\cftbeforesecskip}{0.5em}
\setlength{\cftbeforesubsecskip}{0.25em}
\renewcommand{\contentsname}{Table des matières}

% Configuration des légendes des figures
\usepackage[font=small, it]{caption}

% Commandes personnalisées
\newcommand{\refsec}[1]{\nameref{#1}}


