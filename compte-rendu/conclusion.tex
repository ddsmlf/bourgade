\vspace*{-0.9cm}
\section*{Conclusion}
\addcontentsline{toc}{section}{Conclusion}
\subsection*{Résumé}
Dans ce projet, nous avons mis en place des fonctions pour manipuler des tableaux de ressources et des positions sur un plateau en tentant de prendre en compte l'amélioration continue des fonctionnalités.
Nous avons réalisé des fonctions afin de manipuler différents éléments sur le plateau et une boucle de jeu automatisée.
Nous avons tenté d'optimiser la complexité de chacun de nos algorithmes en priorisant le temps sur l'espace et avons tenté de réaliser des algorithmes de décision optimaux au cours de certains achievements.\\
Finalement, nous avons réalisé un code testé unitairement qui permet à l'utilisateur de choisir la taille et la forme du plateau, le nombre de joueurs et de lancer une partie du jeu Little Town afin d'observer le classement final des différents joueurs.

\subsection*{Axes d'amélioration}
Plusieurs axes pourraient être améliorés sur notre projet pour le rendre plus efficace, plus stable et plus convivial pour l'utilisateur.\\ \\
Nous pourrions ajouter des types de tests plus poussés, comme des tests d'intégration et de bout en bout, pour vérifier que nos fonctions ensemble donnent toujours le résultat attendu.
Nous pourrions aussi ajouter des tests de performance pour s'assurer que l'ajout d'une fonctionnalité dans des conditions de plateau plus grand par exemple reste acceptable pour un ordinateur simple.\\
De plus, il serait intéressant de développer les tests unitaires avant de réaliser les fonctions à tester afin de ne pas être biaisé par le code déjà écrit.\\ \\
En dehors des tests, il pourrait être intéressant de rajouter la possibilité de visualiser la partie "en temps réel" ou encore de laisser de vrais joueurs jouer ensemble ou avec "la machine".\\ \\
Dans ce projet, nous avons valorisé la complexité temporelle en considérant qu'elle était prioritaire sur la complexité spatiale avec les tailles de mémoire élevées des ordinateurs actuels.
Il pourrait cependant être intéressant de chercher de meilleurs compromis pour aussi minimiser la complexité spatiale.\\ \\
Enfin, il serait très utile de rédiger des docstrings pour l'ensemble de nos fichiers sources \texttt{.h}, actuellement réalisées uniquement pour \texttt{decision.h} et \texttt{mine.h}. Ce qui permettrait de revenir plus facilement sur le code déjà réalisé par la suite ou collaborer avec d'autres personnes.

\subsection*{Compétences mises en œuvre} 
Ce projet nous a permis de travailler sur l'étude d'un code implémenté en partie pour le poursuivre sans interférer avec ses fonctionnalités présentes.
Il nous a permis d'améliorer nos méthodes de résolution de bugs et d'organisation du code, d'améliorer notre compréhension du langage \texttt{C} et travailler sur la complexité des algorithmes.
